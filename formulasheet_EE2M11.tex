\documentclass[12pt]{article}
\usepackage[dvipsnames,table]{xcolor}
\usepackage{amsmath}
\usepackage{bbding}
\usepackage{amssymb}
\usepackage{makecell}
\usepackage[a4paper, margin=0.7in, bottom=25mm, top=16mm, headsep=0mm]{geometry}
\usepackage{import}
\usepackage{pdfpages}
\usepackage{transparent}
\usepackage{xcolor}
\usepackage{minted}
\usepackage{hyperref}
\usepackage{longtable}

\newcommand{\nosize}[1]{\raisebox{0pt}[0pt][0pt]{\makebox[0pt]{#1}}}
\newcommand{\nosizep}[1]{\makebox[0pt]{#1}}

\newcommand{\incfig}[2][1]{%
    \def\svgwidth{#1\columnwidth}
    \import{./figures/}{#2.pdf_tex}
}

\newcommand{\icol}[1]{% inline column vector
  \begin{bmatrix}#1\end{bmatrix}%
}

\newenvironment{amatrix}[1]{%
	\left\[\begin{array}{@{}*{#1}{c}|c@{}}
}{%
  \end{array}\right\]
}

\usepackage{xcolor}
\hypersetup{
    colorlinks,
    linkcolor={red!50!black},
    citecolor={blue!50!black},
    urlcolor={blue!80!black}
}

\def\arraystretch{1.3}
%\pdfsuppresswarningpagegroup=1

\title{Formula Sheet EE2M11}
\author{MaybE\_Tree}
\date{2022-09-07}

\begin{document}
\maketitle
%\tableofcontents
%\vfill
%\begin{center}
%	\textit{
%	}
%\end{center}
%\pagebreak

\begin{longtable}{lll}
	\makecell[l]
	{
		Triangle Inequality 
	} &
	\makecell[l]
	{
		$
		\begin{cases}
			|z_1 \pm z_2| \leq |z_1|+|z_2| \\
			|z_1 \pm z_2| \geq |z_1|-|z_2| \\
		\end{cases}
		$
	} &
	\textit{\makecell[l]
		{
	}} \\

	\makecell[l]
	{
		Limits to Infinity
	} &
	\makecell[l]
	{
		$
		\begin{cases}
			\lim\limits_{z \to z_0} f(z) = \infty \iff \lim\limits_{z \to z_0} \cfrac{1}{f(z)} = 0\\
			\lim\limits_{z \to \infty} f(z) = L \iff \lim\limits_{z \to 0} f \left(\cfrac{1}{z}\right) = L\\
		\end{cases}
		$
	} &
	\textit{\makecell[l]
		{
		L must be finite, maybe??
	}} \\

	\makecell[l]
	{
		Cauchy-Riemann
	} &
	\makecell[l]
	{
		\vspace{-3mm}\\
		\begin{tabular}{cccc}
			CR1 & u && v 
			\vspace{2mm}\\
			x &
			$ \cfrac{du}{dx} $ & &
			$ \cfrac{dv}{dx} $ \\
			&&
			\nosize{\rotatebox[origin=c]{-45}{\color{red}$==$}}%
			\nosize{\rotatebox[origin=c]{45}{\color{blue}$-=$}}%
			&\\
			y &
			$ \cfrac{du}{dy} $  & &
			$ \cfrac{dv}{dy} $ \\
		\end{tabular}
	} &
	\textit{\makecell[l]
		{
	}} \\

	\makecell[l]
	{
	} &
	\makecell[l]
	{
		$
		$
	} &
	\textit{\makecell[l]
		{
	}} \\

	\makecell[l]
	{
	} &
	\makecell[l]
	{
		$
		$
	} &
	\textit{\makecell[l]
		{
	}} \\

	\makecell[l]
	{
	} &
	\makecell[l]
	{
		$
		$
	} &
	\textit{\makecell[l]
		{
	}} \\

	\makecell[l]
	{
	} &
	\makecell[l]
	{
		$
		$
	} &
	\textit{\makecell[l]
		{
	}} \\

	\makecell[l]
	{
	} &
	\makecell[l]
	{
		$
		$
	} &
	\textit{\makecell[l]
		{
	}} \\

	\makecell[l]
	{
	} &
	\makecell[l]
	{
		$
		$
	} &
	\textit{\makecell[l]
		{
	}} \\

	\makecell[l]
	{
	} &
	\makecell[l]
	{
		$
		$
	} &
	\textit{\makecell[l]
		{
	}} \\

	\makecell[l]
	{
	} &
	\makecell[l]
	{
		$
		$
	} &
	\textit{\makecell[l]
		{
	}} \\

	\makecell[l]
	{
	} &
	\makecell[l]
	{
		$
		$
	} &
	\textit{\makecell[l]
		{
	}} \\

	\makecell[l]
	{
	} &
	\makecell[l]
	{
		$
		$
	} &
	\textit{\makecell[l]
		{
	}} \\

	\makecell[l]
	{
	} &
	\makecell[l]
	{
		$
		$
	} &
	\textit{\makecell[l]
		{
	}} \\
\end{longtable}


\end{document}

