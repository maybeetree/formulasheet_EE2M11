\documentclass[12pt]{article}
\usepackage[dvipsnames,table]{xcolor}
\usepackage{amsmath}
\usepackage{bbding}
\usepackage{amssymb}
\usepackage{makecell}
\usepackage[a4paper, margin=0.7in, bottom=25mm, top=16mm, headsep=0mm]{geometry}
\usepackage{import}
\usepackage{pdfpages}
\usepackage{transparent}
\usepackage{xcolor}
\usepackage{minted}
\usepackage{hyperref}
\usepackage{longtable}
\usepackage{units}

% small log already defined
\newcommand{\Log}{\operatorname{Log}}
% small arg already defined
\newcommand{\Arg}{\operatorname{Arg}}
\renewcommand{\Re}{\operatorname{Re}}
\renewcommand{\Im}{\operatorname{Im}}
\newcommand{\R}{\mathbb R}
\newcommand{\C}{\mathbb C}
\newcommand{\Z}{\mathbb Z}

\newcommand{\nosize}[1]{\raisebox{0pt}[0pt][0pt]{\makebox[0pt]{#1}}}
\newcommand{\nosizep}[1]{\makebox[0pt]{#1}}

\newcommand{\incfig}[2][1]{%
    \def\svgwidth{#1\columnwidth}
    \import{./figures/}{#2.pdf_tex}
}

\newcommand{\icol}[1]{% inline column vector
  \begin{bmatrix}#1\end{bmatrix}%
}

\newenvironment{amatrix}[1]{%
	\left\[\begin{array}{@{}*{#1}{c}|c@{}}
}{%
  \end{array}\right\]
}

\usepackage{xcolor}
\hypersetup{
    colorlinks,
    linkcolor={red!50!black},
    citecolor={blue!50!black},
    urlcolor={blue!80!black}
}

\def\arraystretch{1.3}
%\pdfsuppresswarningpagegroup=1

\title{Formula Sheet EE2M11}
\author{MaybE\_Tree and Blazico}
\date{2022-09-07}

\begin{document}
\maketitle
%\tableofcontents
%\vfill
%\begin{center}
%	\textit{
%	}
%\end{center}
%\pagebreak

\section{Part 1}

\begin{longtable}{lll}
	\makecell[l]
	{
		Principal \\
		Argument
	} &
	\makecell[l]
	{
		$
		-\pi < \text{Arg}(z) \leq \pi
		$
	} &
	\textit{\makecell[l]
		{
	}} \\

	\makecell[l]
	{
		Triangle \\
		Inequality 
	} &
	\makecell[l]
	{
		$
		\begin{cases}
			|z_1 \pm z_2| \leq |z_1|+|z_2| \\
			|z_1 \pm z_2| \geq |z_1|-|z_2| \\
		\end{cases}
		$
	} &
	\textit{\makecell[l]
		{
	}} \\

	\makecell[l]
	{
		Limits to \\
		Infinity
	} &
	\makecell[l]
	{
		$
		\begin{cases}
			\lim\limits_{z \to z_0} f(z) = \infty \iff \lim\limits_{z \to z_0} \cfrac{1}{f(z)} = 0\\
			\lim\limits_{z \to \infty} f(z) = L \iff \lim\limits_{z \to 0} f \left(\cfrac{1}{z}\right) = L\\
		\end{cases}
		$
	} &
	\textit{\makecell[l]
		{
		L must be finite, maybe??
	}} \\

	\makecell[l]
	{
		Cauchy-\\
		Riemann
	} &
	\makecell[l]
	{
		\vspace{-3mm}\\
		\begin{tabular}{cccc}
			CR1 & u && v 
			\vspace{2mm}\\
			x &
			$ \cfrac{du}{dx} $ & &
			$ \cfrac{dv}{dx} $ \\
			&&
			\nosize{\rotatebox[origin=c]{-45}{\color{red}$==$}}%
			\nosize{\rotatebox[origin=c]{45}{\color{blue}$-=$}}%
			&\\
			y &
			$ \cfrac{du}{dy} $  & &
			$ \cfrac{dv}{dy} $ \\
		\end{tabular}
	} &
	\textit{\makecell[l]
		{
			For \\
			$f(x,y) = u(x+y) + i v(x, y)$ \\
	}} \\

	\makecell[l]
	{
		Harmonic \\
		Check
	} &
	\makecell[l]
	{
		$
			\cfrac{\delta^2 u}{\delta x^2} +
			\cfrac{\delta^2 u}{\delta y^2} = 0
			\implies
			\cfrac{\iint\limits_D f(z)}{\text{Area}(D)}= f(z_c)
		$ 
	} &
	\textit{\makecell[l]
		{
			For \\
			$f(x,y) = u(x+y) + i v(x, y)$ \\
			Around a circular domain $D$ \\
			with centerpointn $ z_c $ 
	}} \\

	\makecell[l]
	{
		Exponential
	} &
	\makecell[l]
	{
		$
		e^z = e^x ( \cos y + i \sin y)
		$
	} &
	\textit{\makecell[l]
		{
	}} \\

	\makecell[l]
	{
		Power
	} &
	\makecell[l]
	{
		$
		z^w = e^{w \log(z)}
		$
	} &
	\textit{\makecell[l]
		{
	}} \\

	\makecell[l]
	{
		Log 
	} &
	\makecell[l]
	{
		$
		\Log(z) = \ln  \left| z \right| + i \arg(z)
		$
	} &
	\textit{\makecell[l]
		{
			$-\pi < \text{Arg}(z) < \pi$
	}} \\

	\makecell[l]
	{
		Trig \\
		Functions
	} &
	\makecell[l]
	{
		$
		\begin{cases}
			\sin z = \cfrac{1}{2i}  \left( e^{iz} - e^{-iz} \right) \\
			\cos z = \cfrac{1}{2}  \left( e^{iz} + e^{-iz} \right) \\
			\sinh z = \cfrac{1}{2}  \left( e^{z} - z^{-z} \right) = -i \sin\left(iz \right) \\
			\cosh z = \cfrac{1}{2}  \left( e^z + e^{-z} \right) = \cos\left(iz \right) \\
		\end{cases}
		$
	} &
	\textit{\makecell[l]{
			Range of $ \sin$ and $\cos$ \\
			is not longer
		$ \pm 1 $ .
	}} \\

	\makecell[l]
	{
		Zakarian \\
		Identities
	} &
	\makecell[l]
	{
		$
		\begin{cases}
			\sin z = \sin x \cosh y + i \cos x \sinh y \\
			\cos z = \cos x \cosh y - i \sin x \sinh y \\
		\end{cases}
		$
	} &
	\textit{\makecell[l]{
	}} \\

	\makecell[l]
	{
		Highschool \\
		Identities
	} &
	\makecell[l]
	{
		$
		\begin{cases}
			\sin\left(z+w \right) =\sin z \cos w + \cos z \sin w \\
			\cos\left(z+w \right) = \cos z \cos w - \sin z \sin w \\
		\end{cases}
		$
	} &
	\textit{\makecell[l]{
	}} \\

	\makecell[l]
	{
		Cauchy's \\
		First Formula
	} &
	\makecell[l]
	{
		$
			2\pi i  f(z_0) =
			\int\limits_C \cfrac{f(z)}{(z-z_0)} 
		$
	} &
	\textit{\makecell[l]
		{
	}} \\

	\makecell[l]
	{
		Cauchy's \\
		Second Formula
	} &
	\makecell[l]
	{
		$
		\cfrac{2\pi i}{n!}  f^{[n]}(z_0) =
			\int\limits_C \cfrac{f(z)}{(z-z_0)^{n+1}} 
		$
	} &
	\textit{\makecell[l]
		{
			For
			$ n = 0, 1, 2, ... $ 
	}} \\

	\makecell[l]
	{
		Parama
	} &
	\makecell[l]
	{
		$
		\int\limits_C f(z) dz = \int\limits_a^b f(z(t)) z'(t) dt
		$
	} &
	\textit{\makecell[l]
		{
	}} \\
\end{longtable}

\section{Part 2}
\begin{longtable}{lll}
	\makecell[l]
	{
		Taylor Theorem
	} &
	\makecell[l]
	{
		$
		\begin{cases}
			f(z) = \sum_{n=0}^{\infty}a_n(z-z_0)^n \\
			a_n = \cfrac{f^{[n]}(z_0)}{n!} \\
		\end{cases}
		$
	} &
	\textit{\makecell[l]
		{
	}} \\

	\makecell[l]
	{
		Binomial \\
		Expansion
	} &
	\makecell[l]
	{
		$
		z^n + n = z_1 \cdot z_2 \cdot z_3 \cdot...
		$
	} &
	\makecell[l]
		{
		$ z_1 $ represents the\\
		$n$th root of $ \sqrt[n]{z^n} $ 
	} \\

	\makecell[l]
	{
	} &
	\makecell[l]
	{
		$
		$
	} &
	\textit{\makecell[l]
		{
	}} \\

	\makecell[l]
	{
	} &
	\makecell[l]
	{
		$
		$
	} &
	\textit{\makecell[l]
		{
	}} \\

	\makecell[l]
	{
	} &
	\makecell[l]
	{
		$
		$
	} &
	\textit{\makecell[l]
		{
	}} \\
\end{longtable}


\begin{tabular}{c|ccccc}
	&
	$0$ &
	$\nicefrac{1}{6} \pi$ &
	$\nicefrac{1}{4} \pi$ &
	$\nicefrac{1}{3} \pi$ &
	$\nicefrac{1}{2} \pi$ \\
	\hline

	$\sin$ &
	$0$ &
	$ \nicefrac{1}{2} $ &
	$ \nicefrac{\sqrt{2}}{2} $ &
	$ \nicefrac{\sqrt{3}}{2} $ &
	$1$ \\

	$\cos$ &
	$1$ &
	$ \nicefrac{\sqrt{3}}{2} $ &
	$ \nicefrac{\sqrt{2}}{2} $ &
	$ \nicefrac{1}{2} $ &
	$0$ \\
	
	$\tan$ &
	$0$ &
	$ \nicefrac{\sqrt{3}}{3} $ &
	$ 1 $ &
	$ \sqrt{3} $ &
	??? \\
	
\end{tabular}


\end{document}

